% Define document class
\documentclass[twocolumn]{aastex631}
\usepackage{showyourwork}

% Begin!
\begin{document}

% Title
\title{Breaking resonances with Post-Main Sequence Winds}

% Author list
\author{Ted Johnson}
\affiliation{University of Nevada, Las Vegas 4505 South Maryland Parkway, Las Vegas, NV 89154, USA}

\author{Rebecca Martin}
\affiliation{University of Nevada, Las Vegas 4505 South Maryland Parkway, Las Vegas, NV 89154, USA}

\author{Stephen Lepp}
\affiliation{University of Nevada, Las Vegas 4505 South Maryland Parkway, Las Vegas, NV 89154, USA}


% Abstract with filler text
\begin{abstract}
    We explore the effects of massive stellar winds on the resonances
    of planets as a host star transitions into a white dwarf.
\end{abstract}

% Main body with filler text
\section{Introduction}
\label{sec:intro}

It has become clear over the past two decades that pollution by metals
of the atmospheres of solitaty white dwarf stars (WDs) can be explained
via accretion of material from it's planetary system (ref Zuckerman+03, Jura08, Klein+10, etc). 
Additionally, the $\nu_6$ resonance has been shown to be an
effective mechanism for delivering rocky asteroids to within
the WD's tidal radius \citep{smallwood2021}.

However, there are now three examples of WD's that are accreting ice-rich
material: G200-39, GD 378, and G238-44 \citep[][, respectively]{xu2017,klein2021,johnson2022}.
Of these three, G200-39 and G238-44 are believed to be the result accretion
of exo-Kuiper belt objects (exo-KBOs) (GD 378 is best explained by an icy moon
of a planet with a strong magnetic field, see \citet{doyle2021}).

The delivery mechanism of exo-KBOs to a WD remains an open question. 
\citet{bonsor2011} investigated the ability of a single planet to scatter
KBOs in towards the host star.

\citet{chen2019} suggested the $\nu_8$ resonance as a mechanism to
deliver water to early Earth. It seems that this could also plausibly
explain pollution by KBOs, except that objects in this resonance are
not expected to survive for a star's main-sequence lifetime. Instead,
a WD system would require that objects are put into this resonance during
the star's transition from main-sequence to white dwarf.

In the solar system, the largest population of KBOs to investigate as
candidates for the $\nu_8$ resonance are the plutinos, or objects in
a 3:2 mean-motion resonance with Neptune. The 3:2 resonance is very stable
and these objects are expected survive for the full
lifetime of the host star \citep{nesvorny2000}.

Given this stability, the 3:2 resonance is not expected to
break provided that mass loss is sufficiently slow (i.e. the
mass loss timescale is much longer than the orbital periods of
the objects). However, in this study we investigate the additional
effects of massive stellar winds that are the result of mass loss.
Effectively, the central mass ``seen'' by an object orbiting a star
shedding it's outer layers is the mass of that star plus the mass of the
wind interior to the object. Effectively, the wind imposes a
position-dependent time delay on the central mass. We investigate
the effects of this differential potential on the resonances of
objects in a 3:2 mean-motion resonance with Neptune.


\section{Methods}
\label{sec:methods}

\subsection{The central force of an isotropic massive wind}

Let $M(t)$ be the mass of the central object. We will assume
that change in $M$ be due to an isotropic wind with some speed
$v(r,t)$

Let $t_{\text{wind}}(r,t)$ be the travel time for wind that arrives to
a radius $r$ at time $t$. For a planet at $r$ the mass interior
to its orbit is $M(t-t_{\text{wind}})$.

In this study we will assume that mass loss occurs exponentially
with an e-folding time of $\tau$. If mass loss begins at $t=0$
then
\begin{equation}
    M = M_0 e^{-t/\tau} \label{eq:m}
\end{equation}

and 

\begin{equation}
    \dot{M} = \frac{-1}{\tau} M_0 e^{-t/\tau} \label{eq:mdot}
\end{equation}

We will also assume that the wind travels at its escape velocity
at all times. For a particle that leaves the star at a time
$t_0$

\begin{equation}
    v^2 = \frac{2 G M(t_0)}{r}
\end{equation}

For $M=M_\odot$ this corresponds to 617 km/s at 1 $R_\odot$
and 7.7 km/s at 30 AU.

Recognizing that $v = \frac{\text{d}r}{\text{d}t}$ and that
$r=R$ at $t=t_0$, we have
\begin{equation}
    r^{\frac{3}{2}} - R^{\frac{3}{2}} = \frac{3}{2}
    \sqrt{2 G M(t_0)} (t-t_0)
\end{equation}

or

\begin{equation}
    t_{\text{wind}} = \frac{2 ( r^{\frac{3}{2}} - R^{\frac{3}{2}} )}
    {3\sqrt{2 G M(t - t_{\text{wind}})}} \label{eq:twind}
\end{equation}

We now define $M_{\text{eff}}(r,t)$ as the mass ``seen'' by a planet at
radius $r$ at time $t$.
\begin{equation}
    M_{\text{eff}}(r,t) = M(t-t_{\text{wind}}) \label{eq:meff}
\end{equation}

If $M$ changes exponentially as in Equation \ref{eq:m} then
\begin{equation}
    M_{\text{eff}}(r,t) = M_0 e^{-(t-t_{\text{wind}})/\tau} \label{eq:meff2}
\end{equation}

Our expression for $t_{\text{wind}}$ then becomes
\begin{equation}
    t_{\text{wind}} = \frac{2 ( r^{\frac{3}{2}} - R^{\frac{3}{2}} )}
    {3\sqrt{2 G M_0 }} \label{eq:twind2}
\end{equation}

Appendix \ref{app:twind} shows the results of solving for
$t_{\text{wind}}$ in Equation \ref{eq:twind2} is

\begin{equation}
    t_{\text{wind}} = 2\tau
    W\left(
        \frac{r^{\frac{3}{2}} - R^{\frac{3}{2}}}
        {3\tau\sqrt{2 G M_0}}
        + e^{t/2\tau}
    \right)
    \label{eq:twindfinal}
\end{equation}

Where $W$ is the Lambert W function.

\bibliography{bib}

\appendix

\section{Solving for $t_{\text{wind}}$}
\label{app:twind}

We now solve for $t_{\text{wind}}$ in Equation \ref{eq:twind2}.
Moving the exponential from the denominator we see

\begin{equation}
    \frac{t_{\text{wind}}}{2\tau} e^{-(t-t_{\text{wind}})/2\tau}
    = \frac{2}{3}
    \frac{r^{\frac{3}{2}} - R^{\frac{3}{2}}}
    {2\tau\sqrt{2 G M_0}}
    \label{eq:twind3}
\end{equation}

Where the factor of $1/2\tau$ has been introduced to make each side
dimensionless.
Taking the natural logarithm of both sides of Equation \ref{eq:twind3} 
and rearranging, we have
\begin{equation}
    \ln{(t_{\text{wind}}/2\tau)}
    + \frac{t_{\text{wind}}}{2\tau}
    = \ln{(A(r))} + \frac{t}{2\tau}
\end{equation}

Where $A(r)$ is the right hand side of Equation \ref{eq:twind3}.
This can easily be put into the form $xe^{x} = b$
where $x = t_{\text{wind}}/2\tau$
and
\[ b = \frac{r^{\frac{3}{2}} - R^{\frac{3}{2}}}
{3\tau\sqrt{2 G M_0}} + e^{t/2\tau} \]

The solution to this equation is $x = W(b)$ where $W$ is
the Lambert W function.
We see then that we can solve for $t_{\text{wind}}$ as
a function of $r$ and $t$:

\begin{equation}
    t_{\text{wind}} = 2\tau
    W\left(
        \frac{r^{\frac{3}{2}} - R^{\frac{3}{2}}}
        {3\tau\sqrt{2 G M_0}}
        + e^{t/2\tau}
    \right)
\end{equation}




\end{document}
